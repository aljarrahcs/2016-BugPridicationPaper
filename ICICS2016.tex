\documentclass[conference]{IEEEtran}
\usepackage{amsfonts,epsfig}
\usepackage{rotating}
\usepackage{algorithm}
\usepackage{algorithmic}
\usepackage{color}
\usepackage{multirow}
\usepackage{graphicx,times,psfig,amsmath} % Add all your packages here
\usepackage{flushend}
\usepackage{amssymb,amsfonts,amsbsy,epsfig,psfig}
%\usepackage[noend]{algpseudocode}
%\usepackage{times}
%\usepackage{multirow}
%\usepackage{latexsym}
%\usepackage{textcomp}
%\usepackage{epstopdf}
%\usepackage[latin9]{inputenc}
%\usepackage[T1]{fontenc}
%\usepackage{babel}
%\usepackage{amssymb}
%\usepackage{amsthm}

\hyphenation{op-tical net-works semi-conduc-tor IEEEtran}
\usepackage{cite}
\IEEEoverridecommandlockouts    % to create the author's affiliation portion % using \thanks
\def\citedash{]--[}

%------------------------------------------------------------
\begin{document}

\title{Bug Prediction Title}

\author{\IEEEauthorblockN{ ,  ,  }
\IEEEauthorblockA{\\ \\
The University of Jordan\\
Amman, Jordan\\
\{ ,  ,  \} }\\

}
%--------------------------------------------------------
% make the title area ---------------------------------
%--------------------------------------------------------
\maketitle

\begin{abstract}

\end{abstract}
\begin{IEEEkeywords} Krill Herd, Genetic Algorithm, Feedforward Neural Network, Optimization, Spam. \end{IEEEkeywords}
\IEEEpeerreviewmaketitle
\bibliographystyle{ieeetr}

%---------------------------------------------------------------------------
\section{Introduction and Related work}
 To Do To Do
 To Do To Do
%-------------------------------------------------------
\section{Feedforward Neural Networks}
\label{nn}
 To Do To Do
 To Do To Do

\section{Proposed Methodology}
 To Do To Do
 To Do To Do

\section{Experiments and results}
 To Do To Do
 To Do To Do

\subsection{Environment}
 To Do To Do
 To Do To Do

\subsection{Dataset preprocessing}
 To Do To Do
 To Do To Do
\subsection{Experiments setup}
 To Do To Do
 To Do To Do

\subsection{Evaluation Measures}
 In order to evaluate the performance of the developed classification models, four different ratio are used: the accuracy rate, the Area Under ROC  (AUC), The TP rate and the FP rate. The four measurements can be described as follows:
 
 % Add a table for confusion matrix 
 
 To Do To Do
 To Do To Do
\subsection{Results and discussion}
 To Do To Do
 To Do To Do

%%%%%%%%%%%%%%%%%%%%%%%%%%%%%%%%%%%%%%%%%%%%%%%%%%%%%%%%%%%%%%%%%%%%%%%
%
%           Conclusion and Future Work
%
%%%%%%%%%%%%%%%%%%%%%%%%%%%%%%%%%%%%%%%%%%%%%%%%%%%%%%%%%%%%%%%%%%%%%%%
\section{Conclusions}
 To Do To Do
 To Do To Do


\baselineskip 2.4ex 
\bibliography{ref}
\end{document}

